\chapter{Magnetorquer characteristics} \label{chap:F}
The characteristics of the magnetorquers used in the satellite is inspired by \cite{TH}, where two types of magnetorquers are described: one with metal core and one without core. Having an iron core increases controllable magnetic moment change at the expense of control accuracy.  The parameters for one magnetorquer without core are presented in table \ref{table:for}:

\begin{table}[H]
	\centering
	\begin{tabular}{|l|l|}
		\hline
		\textit{\textbf{Parameter}}     & \textit{\textbf{Value}}                     				     \\ \hline
		Coil size                       		   & 75x75 {[}$mm^2$ {]}                      					  \\ \hline
		Wire Thickness                      & 0.13 {[}$mm${]}                             					 	\\ \hline
		Windings                        		& 250                                          						     	 \\ \hline
		Coil mass                     		    & 0.053 [$kg$]                                    					    \\ \hline
		Max voltage                  	      & ±1.25 {[}$V${]}  		\\ \hline
		Max current                  	      & 15.78 {[}$mA${]}                               						   \\ \hline
		Actuator on time                   & 88\%                                      				                        \\ \hline
		Max power consumption pr. coil  & 17.4 {[}$mW$ {]}                            					     \\ \hline
		Total power consumption     & 134.2 {[}$mW${]}                           	  					   	    \\ \hline
		Coil Discharge time:             & 0.33481 {[}$ms${]} (99\% discharged)   					     \\ \hline
		Available time for measurements & 11.67 {[}$ms${]}                               					     \\ \hline
	\end{tabular}
	\caption{Parameters for magnetorqer without metal core}
	\label{table:for}
\end{table}

For this type of magnetorqer, one magnetorqer will generate around 200 [$nNm$] at low magnetic field strength (18000 [$nT$), which is perpendicular to the area of the coil. (thesis)

A second alternative is to choose a magnetorqer with metal core, because the power consumption, the size and weight are considered superior compared with a magnetorqer without cores. Moreover, because the interest in redundancy is important and four magnetorqers will be placed inside the satellite, the magnetorqer with metal core is preferable because of their weight and size. The parameters for the magnetorqer with metal core is illustrated in the following table:
\begin{table}[H]
	\centering
	\begin{tabular}{|l|l|}
		\hline
		\textit{\textbf{Parameter}}     & \textit{\textbf{Value}}               \\ \hline
		Core diameter:                       & 10 {[}$mm$ {]}                           \\ \hline
		Core length             			   & 10 {[}$mm${]}                             \\ \hline
		Permeability                           & 1000                                      	     \\ \hline
		Wire Thickness                      & 0.13 [$mm$]                                 \\ \hline
		Windings                                & 200 											      \\ \hline
		Coil mass                               & 0.019 {[}$kg${]}                               \\ \hline
		Max voltage                			  & ±1.25 {[}$V${]}                                  \\ \hline
		Max current							  & 20 {[}$mA$ {]}                                    \\ \hline
		External resistance needed   & 62.5 {[}$\Omega$ {]} 					  	 \\ \hline
		Actuator on time			       & 27 \% {[}$mW${]}                                  \\ \hline
		Max power consumption pr. coil     & 6.75{[}$mW${]} 				           \\ \hline
		Total power consumption		& 35.3 {[}$mW${]}                                      \\ \hline
	\end{tabular}
	\caption{Parameters for magnetorqer with metal core}
	\label{table:for1}
\end{table}
In this case, one magnetorqer will generate around 400 [$nNm$] at low magnetic field strength (18000 $[nT]$). Besides the advantage of having a reduced dimension and a better power consumption, the magnetorqer with core, generates a magnetic moment higher than expected. On the other hand, the magnetorquer without core was already tested on the AAUSAT and can be seen as a safe choice of actuator.


