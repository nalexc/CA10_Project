\chapter{Derivation of equation of motion } \label{chap:A}
The general Euler's rotation equation with three reaction wheels aligned on the satellite body axis are derived as
%
\begin{flalign}
	{I_{1}  {\dot \omega_{1}}} ={(I_{2}-I_{3})\omega_{2}\omega_{3}+N_{1}-\omega_{2}h_{3}+\omega_{3}h_{2}}
	\label{eq:angularmomentum2Appedix1}
\end{flalign}
%
\begin{flalign}
	{I_{2}  {\dot \omega_{2}}} ={(I_{3}-I_{1})\omega_{1}\omega_{3}+N_{2}-\omega_{3}h_{1}+\omega_{1}h_{3}}
	\label{eq:angularmomentum2Appedix2}
\end{flalign}  
%
\begin{flalign}
	{I_{3}  {\dot \omega_{3}}} ={(I_{1}-I_{2})\omega_{1}\omega_{2}+N_{3}-\omega_{1}h_{2}+\omega_{2}h_{1}}
	\label{eq:angularmomentum2Appedix3}
\end{flalign}
%
The equation in compact form has been written as 
%
\begin{flalign}
	{\dot{\omega}} ={-\underline I_{s}^{-1}\underline S(\omega)\underline I_{s}^{-1}\vec{\omega}-\underline I_{s}^{-1}\underline S(\omega)\vec {h_{tot}}-\underline I_{s}^{-1}\vec {\dot{h}_{mw}}+\underline I_{s}^{-1}\vec {N_{tot}}}
	\label{eq:angularmomentum2Appedix4}
\end{flalign}
%
where $\underline S(\omega)$ is the skew symmetric matrix given by
%
\begin{flalign}
	{\underline S(\omega)}
	= 
	\begin{bmatrix}
		0& -\omega_{3}& \omega_{2} \\
		\omega_{3}& 0&-\omega_{1}  \\ 
		-\omega_{2} & \omega_{1} &0
	\end{bmatrix} 
	\label{eq:skewsymmetricmatrix}
\end{flalign}
%
and the angular momentum of the reaction wheels as $\vec h_{mw}=[h_1 \ h_2 \ h_3]^{T}$.
\subsection{Inertia matrix}
%
The inertia matrix for a solid cuboid of height $z$ , width $y$, and depth $x$, amd mass $m_{i}$ with respect the center of mass is given by 
%
\begin{flalign}
	\underline {I}_{i}
	= 
	\begin{bmatrix}
		\frac{1}{12} m_i(z^{2}+y^{2}) &0&0 \\
		0&  \frac{1}{12}m_i(z^{2}+x^{2})&0   \\ 
		0 & 0 &\frac{1}{12} m_i(x^{2}+y^{2}
	\end{bmatrix} 
	\label{eq:inertiaTensorMatrix}
\end{flalign}
%
It is assumed that the Cube have a symmetric mass distribution around the axis of rotation to simplify the inertia matrix.  
With the mass distributed evenly and the axis of rotation being around one of the tree axis, the off diagonal term of the inertia matrix are equal to zero. These terms are also referred to as cross products of inertia.