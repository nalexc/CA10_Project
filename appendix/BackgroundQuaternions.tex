\chapter{Quaternions } \label{chap:B}
This appendix is based on sources from \cite{SADC} and \cite{Kui}.

There are several possible mathematical representations for rotation. In physics, rotation matrices, Euler angles (eg. pitch-roll-yaw) and quaternions. In satellite engineering, quaternions are the preferred representations, since they are more compact than rotation matrices and lack singularities.

Quaternions include four values, three of them represent a vector \textbf{$\epsilon$}, the fourth a scalar $\eta$. 
\begin{equation}
\textbf{q} =
\left[ 
\begin{array}{cccc}
q_1 \\
q_2 \\  
q_3 \\
q_4 
\end{array}
\right] 
= 
\left[ 
\begin{array}{cccc}
\textbf{$\epsilon$} \\
\eta
\end{array}
\right] 
\end{equation}
A rotation with $\Phi$ around the unit vector can be described according to Euler's formula.
\begin{equation}
q = e^{\frac{\Phi}{2} (e_1 \textbf{i}+ e_2 \textbf{j} + e_3 \textbf{k} + e_4)} = \cos \frac{\Phi}{2} + (e_1 \textbf{i}+ e_2 \textbf{j} + e_3 \textbf{k} +e_4) \sin \frac{\Phi}{2}
\end{equation}

Consequently 
\begin{equation}
\textbf{q} =
\left[ 
\begin{array}{cccc}
q_1 \\
q_2 \\  
q_3 \\
q_4 
\end{array}
\right] 
= 
\left[ 
\begin{array}{cccc}
e_1  \sin \frac{\Phi}{2} \\
e_2  \sin \frac{\Phi}{2} \\  
e_3  \sin \frac{\Phi}{2} \\
\cos \frac{\Phi}{2} 
\end{array}
\right] 
\end{equation}

\subsection{Quaternion multiplication}
Let \textbf{q} represent the unit length rotation axis, with \textbf{i}, \textbf{j}, \textbf{k} being the base vectors in euclidean space and $e_4$ as the scalar part: 
\begin{equation}
\textbf{q} = q_1 \textbf{i}  + q_2 \textbf{j} + q_3 \textbf{k} + q_4
\end{equation}

where $\textbf{i}, \textbf{j}, \textbf{k}$ represent the hyper imaginary parts and satisfying the rules introduced by Hamilton:
\begin{align*}
	\begin{split}
		\vec {i^{2}} &=  \vec {j^{2}}  = \vec {k^{2}} = -1 \\
		{\vec {i}} {\vec {j}} &= - {\vec {j}} {\vec {i}} = \vec k \\
		{\vec {j}} {\vec {k}} &= - {\vec {k}} {\vec {j}} = \vec i \\
		{\vec {k}} {\vec {i}} &= - {\vec {i}} {\vec {k}} = \vec j 
	\end{split}
\end{align*}
Next a product of two quaternions $\vec q_A$ and $\vec q_B$ is illustrated:
\begin{flalign}
	\vec q_C = \vec q_A \otimes \vec q_B = ( q_{A_{1}} \vec i + q_{A_{2}} \vec j + q_{A_{3}} \vec k + q_{A_{4}}) \otimes ( q_{B_{1}} \vec i + q_{B_{2}} \vec j + q_{B_{3}} \vec k + q_{B_{4}})
		\label{eq:quat}
\end{flalign}
After rearranging terms and using the rules above, equation \ref{eq:quat} becomes:
\begin{flalign}
\vec q_C &= ( q_{A_{1}} q_{B_{4}} + q_{A_{2}}q_{B_{3}} - q_{A_{3}}q_{B_{2}}  + q_{A_{4}} q_{B_{1}}) \vec i + \\ 
&+( - q_{A_{1}} q_{B_{3}} + q_{A_{2}}q_{B_{4}} - q_{A_{3}}q_{B_{1}}  + q_{A_{4}} q_{B_{2}}) \vec j + \\
&+( q_{A_{1}} q_{B_{2}} - q_{A_{2}}q_{B_{1}} - q_{A_{3}}q_{B_{4}}  + q_{A_{4}} q_{B_{3}}) \vec k + \\
&+ (- q_{A_{1}} q_{B_{1}} - q_{A_{2}}q_{B_{2}} - q_{A_{3}}q_{B_{3}}  + q_{A_{4}} q_{B_{4}}) 
	\label{eq:quat2}
\end{flalign}
The product quaternion can be expressed in a matrix form:
\begin{flalign}   
	\begin{bmatrix}
		 q_{C_{1}} \\
		 q_{C_{2}}  \\ 
		 q_{C_{3}}  \\ 
		 q_{C_{4}}  \\ 
	\end{bmatrix} 
	&= 
		\underbrace{
	\begin{bmatrix}
		q_{A_{4}}& q_{A_{3}}& -q_{A_{2}}&q_{A_{1}}&\\
		-q_{A_{3}}& q_{A_{4}}& q_{A_{1}}&q_{A_{2}}&\\
		q_{A_{2}}& -q_{A_{1}}& q_{A_{4}}&q_{A_{3}}&\\
		-q_{A_{1}}& -q_{A_{2}}& -q_{A_{3}}&q_{A_{4}}&\\
	\end{bmatrix} 
}_{\underline{C}_A}
	\begin{bmatrix}
    q_{B_{1}} \\
	q_{B_{2}}  \\ 
	q_{B_{3}}  \\ 
	q_{B_{4}}  \\  
	\end{bmatrix} 	
	\label{eq:sfgdm}
\end{flalign}
A skew-symmetric matrix is given as $\underline{S}(\vec q)^\times$ and defined as
\begin{flalign}   
	\underline{S}(\vec q)^\times = 
	\begin{bmatrix}
		 0 & - q_{A_{3}}  & q_{A_{2}}\\
		 q_{A_{3}}&  0  & - q_{A_{1}} \\ 
		- q_{A_{2}}&  q_{A_{1}}  & 0 \\ 
	\end{bmatrix} 
	\label{eq:s2f}
\end{flalign}

Moreover equation \ref{eq:sfgdm} can be written as
\begin{flalign}   
	\vec q_C = \vec q_A \vec q_B = \underline{C}_A \vec q_B = 
	\begin{bmatrix}
		 - \underline S(\vec{ q})^\times + \underline{\vec 1} q_{C_{4}} & \vec{ q} \\
		 - \vec{ q}^\mathsf{T}&  q_{C_{4}}  \\ 
	\end{bmatrix} 
   \vec q_B
	\label{eq:sff}
\end{flalign}
\subsection{Properties of quaternions}
The \textit{complex conjugate} of a quaternion $\vec q$ is given by
\begin{flalign}
	\vec q^\ast= - q_1 \vec i - q_2 \vec j - q_3 \vec k +q_4
	\label{eq:quart2}
\end{flalign}
Thus
\begin{flalign}
	(\vec q_A \vec q_B) = \vec q^\ast _{B} \vec q^\ast_{A}
	\label{eq:quart32}
\end{flalign}
The \textit{norm} of a quaternion $\vec q$, denoted by $|\vec q|$ is
\begin{flalign}
	|\vec q|= \vec q \vec q^\ast = \vec q^\ast \vec q = |\vec q|^2 = \sqrt{q^2_{1} + q^2_{2} +q^2_{3}+q^2_{4}}
	\label{eq:quar42}
\end{flalign}
The \textit{inverse} of a quaternion $\vec q$ is defined as 
\begin{flalign}
	\vec q^{-1} = \dfrac{\vec q^\ast }{|\vec q|^2}
	\label{eq:quar432}
\end{flalign}
It can be verified that 
\begin{flalign}
	\vec q^{-1} \vec q = \vec q \vec q^{-1} = 1 
	\label{eq:qutt}
\end{flalign}
where $\vec q$ is the unit quaternion and the inverse is its conjugate $\vec q^{-1} $