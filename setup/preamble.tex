\documentclass[12pt,twoside,a4paper]{report}
\usepackage{etex}
% Select encoding of your inputs.
\usepackage[utf8]{inputenc}
\usepackage{nomencl}
% Make latex understand and use the typographic
% rules of the language used in the document.
\usepackage[english]{babel}

% Use the vector font Latin Modern which is going
% to be the default font in latex in the future.
\usepackage{lmodern}

% use package for figure zooming


% Choose the font encoding
\usepackage[T1]{fontenc}

% Use colour in tables
\usepackage[table]{xcolor}
\usepackage{array}
\usepackage{multirow}

% load a colour package
\usepackage{xcolor}
\definecolor{aaublue}{RGB}{33,26,82}% dark blue

% The standard graphics inclusion package
\definecolor{white}{RGB}{255,255,255} % define color white
\usepackage{graphicx}
\usepackage{adjustbox}

% Set up how figure and table captions are displayed
\usepackage{caption}
\captionsetup{
  font=footnotesize,% set font size to footnotesize
  labelfont=bf % bold label (e.g., Figure 3.2) font
}

% Enable row combination in tables
\usepackage{multirow}

% Make space between table lines and text
\renewcommand{\arraystretch}{1.5}

% Enable commands like \st (strike out) and \hl (high light)
\usepackage{soul}

% Make the standard latex tables look so much better
\usepackage{array,booktabs}
\usepackage[normalem]{ulem}

% Enable the use of frames around, e.g., theorems
% The framed package is used in the example environment
\usepackage{framed}
\usepackage{colortbl}
\usepackage{longtable}
\usepackage{xcolor}
\usepackage{textcomp}

%%%%%%%%%%%%%%%%%%%%%%%%%%%%%%%%%%%%%%%%%%%%%%%%
% Mathematics
%%%%%%%%%%%%%%%%%%%%%%%%%%%%%%%%%%%%%%%%%%%%%%%%
% Defines new environments such as equation,
% align and split 
\usepackage{amsmath}
\usepackage{relsize}
% Adds new math symbols
\usepackage{amssymb}
% Use theorems in your document
% The ntheorem package is also used for the example environment
% When using thmmarks, amsmath must be an option as well. Otherwise \eqref doesn't work anymore.
\usepackage[framed,amsmath,thmmarks]{ntheorem}
\usepackage{nomencl}
%%%%%%%%%%%%%%%%%%%%%%%%%%%%%%%%%%%%%%%%%%%%%%%%
% Page Layout
%%%%%%%%%%%%%%%%%%%%%%%%%%%%%%%%%%%%%%%%%%%%%%%%
% Change margins, papersize, etc of the document
\usepackage[
  left=25mm,% left margin on an odd page %tidligere 25mm for baade right og left
  right=25mm,% right margin on an odd page
  top=35mm,
  ]{geometry} % TODO go back to this default margin
%\usepackage[left=4cm,right=4cm,top=3cm,bottom=3cm]{geometry} % giving enough space for todonotes

  
% Modify how \chapter, \section, etc. look
% The titlesec package is very configureable
\usepackage{titlesec}
\makeatletter
\def\ttl@mkchap@i#1#2#3#4#5#6#7{%
    \ttl@assign\@tempskipa#3\relax\beforetitleunit
    \vspace{\@tempskipa}%<<<<<< REMOVE THE * AFTER \vspace
    \global\@afterindenttrue
    \ifcase#5 \global\@afterindentfalse\fi
    \ttl@assign\@tempskipb#4\relax\aftertitleunit
    \ttl@topmode{\@tempskipb}{%
        \ttl@select{#6}{#1}{#2}{#7}}%
    \ttl@finmarks  % Outside the box!
    \@ifundefined{ttlp@#6}{}{\ttlp@write{#6}}}
\makeatother

\titlespacing{\chapter}{0pt}{0pt}{10pt}
\titlespacing{\section}{0pt}{0pt}{-5pt}
\titlespacing{\subsection}{0pt}{8pt}{-5pt}
\titlespacing{\subsubsection}{0pt}{6pt}{-10pt}

\titleformat*{\section}{\normalfont\Large\bfseries\color{aaublue}}
\titleformat*{\subsection}{\normalfont\large\bfseries\color{aaublue}}
\titleformat*{\subsubsection}{\normalfont\normalsize\bfseries\color{aaublue}}

\usepackage{titlesec, blindtext, color}
%\color{gray75}{gray}{0.75}
\newcommand{\hsp}{\hspace{20pt}}
\titleformat{\chapter}[hang]{\Huge\bfseries}{\thechapter\hsp\textcolor{aaublue}{|}\hsp}{0pt}{\Huge\bfseries}

% Change the headers and footers
\usepackage{fancyhdr}
\setlength{\headheight}{15pt}
\pagestyle{fancy}
\fancyhf{} %delete everything
\renewcommand{\headrulewidth}{0pt} %remove the horizontal line in the header
\fancyhead[RO,LE]{\color{aaublue}\small\nouppercase\leftmark} %even page - chapter title
\fancyhead[LO]{}
\fancyhead[RE]{} 
\fancyhead[CE]{}
\fancyhead[CO]{}
\fancyfoot[RE,LO]{\thepage}
%\fancyfoot[LE,RO]{Gr. 834} %page number on all pages
\fancyfoot[CE,CO]{}

% change first page of all chapters header and footer to fancy style
\makeatletter
\let\ps@plain\ps@fancy
\makeatother

% Do not stretch the content of a page. Instead,
% insert white space at the bottom of the page
\raggedbottom

% Enable arithmetics with length. Useful when typesetting the layout.
\usepackage{calc}

%%%%%%%%%%%%%%%%%%%%%%%%%%%%%%%%%%%%%%%%%%%%%%%%
% Bibliography
%%%%%%%%%%%%%%%%%%%%%%%%%%%%%%%%%%%%%%%%%%%%%%%%
%setting references (using numbers) and supporting i.a. Chicargo-style:
%\usepackage[citestyle=authoryear,natbib=true]{biblatex}
\usepackage[backend=biber]{biblatex}
%\usepackage{etex}
%\usepackage{etoolbox}
%\usepackage{keyval}
%\usepackage{ifthen}
%\usepackage{url}
%\usepackage{csquotes}
%\usepackage[backend=biber, url=true, doi=true, style=numeric, sorting=none]{biblatex}
%\addbibresource{setup/bibliography.bib}
\bibliography{setup/bibliography.bib}

\usepackage[utf8]{inputenc}
\usepackage[english]{babel}

\usepackage{comment}

%\usepackage[
%backend=biber,
%style=numeric,
%sorting=ynt
%]{biblatex}
%\addbibresource{setup/bibliography.bib}

%%%%%%%%%%%%%%%%%%%%%%%%%%%%%%%%%%%%%%%%%%%%%%%%
% Misc
%%%%%%%%%%%%%%%%%%%%%%%%%%%%%%%%%%%%%%%%%%%%%%%%

%%% Enables the use FiXme refferences. Syntax: \fxnote{...} %%%
\usepackage[footnote, draft, english, silent, nomargin]{fixme}
%With "final" instead of "draft" an error will ocure for every FiXme under compilation.

%%% allows use of lorem ipsum (generate i.e. pagagraph 1 to 5 with \lipsum[1-5]) %%%
\usepackage{lipsum}

%%% Enables figures with text wrapped tightly around it %%%
\usepackage{wrapfig}

%%% Section debth included in table of contents (1 = down to sections) %%%
\setcounter{tocdepth}{1}

%%% Section debth for numbers (1 = down to sections) %%%
\setcounter{secnumdepth}{1}

\usepackage{tocloft}
\setlength{\cftbeforetoctitleskip}{0 cm}
\renewcommand{\cftpartpresnum}{Part~}
\let\cftoldpartfont\cftpartfont
\renewcommand{\cftpartfont}{\cftoldpartfont\cftpartpresnum}

%%%%%%%%%%%%%%%%%%%%%%%%%%%%%%%%%%%%%%%%%%%%%%%%
% Hyperlinks
%%%%%%%%%%%%%%%%%%%%%%%%%%%%%%%%%%%%%%%%%%%%%%%%

% Enable hyperlinks and insert info into the pdf
% file. Hypperref should be loaded as one of the 
% last packages
\usepackage{nameref}
\usepackage{hyperref}
\hypersetup{%
	%pdfpagelabels=true,%
	plainpages=false,
	pdfauthor={Author(s)},%
	pdftitle={Title},%
	pdfsubject={Subject},%
	bookmarksnumbered=true,%
	colorlinks,%
	citecolor=aaublue,%
	filecolor=aaublue,%
	linkcolor=aaublue,% you should probably change this to black before printing
	urlcolor=aaublue,%
	pdfstartview=FitH%
}

% remove all indentations
\setlength\parindent{0pt}
\parskip 5mm
\usepackage{verbatim}

\definecolor{Gra}{RGB}{230,230,230}

%creates a nice-looking C#-text
\newcommand{\CC}{C\nolinebreak\hspace{-.05em}\raisebox{.3ex}{\scriptsize\text \#} }

%enables multi column lists
\usepackage{multicol}

%enables code-examples
\usepackage{listings}

\definecolor{coolblue}{RGB}{32,95,128}
\definecolor{mygreen}{rgb}{0,0.6,0}
\definecolor{mygray}{rgb}{0.5,0.5,0.5}
\definecolor{mymauve}{rgb}{0.58,0,0.82}
\usepackage{textcomp}
\definecolor{listinggray}{gray}{0.9}
\definecolor{lbcolor}{rgb}{0.9,0.9,0.9}

\lstset{
backgroundcolor=\color{lbcolor},
	tabsize=4,
	rulecolor=,
	language=C,
        basicstyle=\scriptsize,
        upquote=true,
        aboveskip={1.5\baselineskip},
        columns=fixed,
        showstringspaces=false,
        extendedchars=true,
        breaklines=true,
        prebreak = \raisebox{0ex}[0ex][0ex]{\ensuremath{\hookleftarrow}},
        frame=single,
        showtabs=false,
        numbers=left,
        captionpos=b,
        numbersep=5pt,
        numberstyle=\tiny\color{mygray},
        showspaces=false,
        showstringspaces=false,
        identifierstyle=\ttfamily,
        keywordstyle=\color[rgb]{0,0,1},
        commentstyle=\color[rgb]{0.133,0.545,0.133},
        stringstyle=\color[rgb]{0.627,0.126,0.941},
}

%% ADD MATLAB COLOR CODE
\lstdefinestyle{custommatlab}{
	backgroundcolor=\color{lbcolor},
	tabsize=4,
	rulecolor=,
	language=Matlab,
	basicstyle=\scriptsize,
	upquote=true,
	aboveskip={1.5\baselineskip},
	columns=fixed,
	showstringspaces=false,
	extendedchars=true,
	breaklines=true,
	prebreak = \raisebox{0ex}[0ex][0ex]{\ensuremath{\hookleftarrow}},
	frame=single,
	showtabs=false,
	numbers=left,
	captionpos=b,
	numbersep=5pt,
	numberstyle=\tiny\color{mygray},
	showspaces=false,
	showstringspaces=false,
	identifierstyle=\ttfamily,
	keywordstyle=\color[rgb]{0,0,1},
	commentstyle=\color[rgb]{0.133,0.545,0.133},
	stringstyle=\color[rgb]{0.627,0.126,0.941},   
}
\lstdefinestyle{custommatlabinline}{
	style=custommatlab,
	basicstyle=\small,
}

\usepackage{float}
\usepackage{caption}
\usepackage{subcaption}
\usepackage{siunitx}
\sisetup{decimalsymbol=comma}
\sisetup{detect-weight}

\usepackage{enumitem}

% Figures - TIKZ
\usepackage{tikz}
\usetikzlibrary{shapes,arrows}
\usepackage[americanresistors,americaninductors,americancurrents, americanvoltages]{circuitikz}

% Wall of text logo
\newcommand{\walloftextalert}[0]{\includegraphics[width=\textwidth]{walloftext.png}}

\usepackage{pdfpages}
\usepackage{lastpage}
\usepackage{epstopdf}

\setlength{\headheight}{21pt}

\hfuzz=\maxdimen
\tolerance = 10000
\hbadness  = 10000

\usepackage{siunitx}
\graphicspath{{./figures/}}

\usepackage{todonotes} 
\paperwidth=\dimexpr \paperwidth + 10cm\relax
\oddsidemargin=\dimexpr\oddsidemargin + 5cm\relax
\evensidemargin=\dimexpr\evensidemargin + 5cm\relax
\marginparwidth=\dimexpr \marginparwidth + 3cm\relax

\usepackage{nomencl}
\makenomenclature


\usepackage[toc,xindy]{glossaries}
\makeglossaries


\usepackage{ifthen}
\renewcommand{\nomgroup}[1]{%
	\ifthenelse{\equal{#1}{T}}{\item[\textbf{\Large Terminology}]}{%
	\ifthenelse{\equal{#1}{A}}{\item[\textbf{\Large Acronyms}]}{%
	\ifthenelse{\equal{#1}{R}}{\item[\textbf{\Large Reference Frames}]}{%
	\ifthenelse{\equal{#1}{S}}{\item[\textbf{\Large Symbols}]}{}}
}}}

\setlength{\nomlabelwidth}{4.5cm} % set spacing between symbol and description

\renewcommand{\nompreamble}{
	\section*{Notations}

Vectors used have a bold typeface.  
\begin{equation*}
\textbf{v}
\end{equation*}
Matrices are underlined.
\begin{equation*}
\underline{A}
\end{equation*}

Cross product operations can be evaluated by taking the skew symmetric matrix of the left vector and executing a matrix multiplication. The skew symmetric matrix of $\textbf{v}$ is denoted as $\underline{v}^\times$
\begin{equation*}
	\textbf{w} = \textbf{u} \times \textbf{v} = \underline{u}^\times \textbf{v}
\end{equation*}
Matrix transposition is denoted as
\begin{equation*}
\underline{A}^T
\end{equation*}
If a non-square matrix has to undergo an operation similar to inversion, Moore-Penrose pseudoinverse is used. Pseudoinverse matrix is indicated as $\underline{A}^\dagger$. If the matrix satisfies $rank(\underline{A}) = min(m,n)$ and $m < n$, the left pseudoinverse is used as follows
\begin{equation*}
\underline{A}^\dagger    =   (\underline{A}^T \underline{A} )^{-1} \underline{A}^T 
\end{equation*}

If $n < m$, the right pseudoinverse is used as follows

\begin{equation*}
 \underline{A}^\dagger    =  \underline{A}^T  (\underline{A} \underline{A}^T)^{-1}
\end{equation*}


The majority of equations are expressed in body-fixed frame (BFF). Unless it's not explicitly noted, the matrices and vectors are expressed in BFF. In case the expression is in earth centered inertial frame (ECI), it is noted in the superscript as 

\begin{equation*}
\vec{XY}^{[I]}
\end{equation*}

The rotation quaternion between frames use the subscript to denote the frame where the transformation is done from, the superscript is the symbol of the frame being transformed into. In case the frame symbols are not present, it should be interpreted as a transformation from inertial frame to body frame.

\begin{equation*}
\vec{^s_i q(t)}
\end{equation*}

Rotation matrix corresponding to rotation matrix $\vec{^s_i q(t)}$ is denoted as
\begin{equation*}
\underline{R}(\vec{^s_i q(t)})
\end{equation*}}