\chapter{System Description}\label{chap:systemDescribtion}

\section{Reaction Wheels}

One method of controlling a spacecraft's attitude is by using either reaction or momentum wheels attached to the spacecraft's body. By controlling the wheel's angular velocity using a motor, the amount of angular momentum stored in the wheel can be controlled. If there are no external forces involved, the sum of angular momentum in the system made up by the spacecraft's body and the reaction wheels is constant. This means that by increasing the angular velocity of the wheels, the satellite body's angular momentum can be reduced. This angular momentum transfer can be used to control the attitude of the satellite. If the goal is to change the angular momentum of the whole satellite, actuators that have external interaction should be used, such as magnetorquers or solar sails.

The difference between momentum and reaction wheels is that the nominal angular velocity of momentum wheels is high in order to store angular momentum, and in the case of reaction wheels, low. Many momentum wheels still turn at an angular velocity larger than zero in order to avoid static friction in the bearings. Reaction wheels usually make up only a small fraction of a satellite's weight. They rely on being able to run at high speeds, making their angular momentum significant. The small weight ratio makes precise controlling easier.


%One design consideration for reaction wheels is maximizing moment of inertia for unit weight. This is done by distributing most of the material near the outskirts of the wheel. There is a trade-off between having most of the mass at the outskirts and durability at high angular velocities. 


Reaction wheels have an angular velocity limitation. This means that if a reaction wheel reaches its maximum angular velocity, it can no longer generate a torque on the satellite's body in one direction. In this scenario the system's controllability decreases, thus it should be avoided. An angular momentum unloading strategy should be designed to avoid it. Instead of returning the angular momentum to the satellite's body, unloading the angular momentum through other methods is preferred. Magnetorquers can be used for such purposes.

Moving parts are usually prone to failures. Reaction wheels are expected to run at high angular velocities, which wears down the lubrication and the bearings. Reaction wheels equipped with active magnetic bearings are in development \cite{MagneticReactWheel}. These can eliminate friction from the system and by controlling the bearing, can even reduce micro-vibrations, increasing the durability of the system. AAUSAT-II itself however uses mechanical wheel bearings.

\subsection{Reaction Wheel Configuration}

There have been studies on what is the best configuration of redundant reaction wheels. The optimal configuration can of course depend on the requirements. If the requirement is to have the same controllability for reaction wheels in case of fault, and 6 reaction wheels are available, orthogonally configured double reaction wheels can be used. Minimizing energy consumption is normally the goal in deciding on a configuration. Ismail et al. \cite{ReactionWheelConfigSim} investigated several configurations by running simulations with the configuration being the only difference. The tetrahedron configuration of 4 reaction wheels has been chosen, which is quite widespread in the industry \cite{reactConfigNasa}.
In tetrahedron configuration the 4 wheel orientations are evenly distributed, unlike the also widespread pyramid configuration. 

If the requirement for the satellite is tracking objects on Earth, the tracking torque demand can be calculated using knowledge about satellite altitude, orbit shape and the corresponding satellite speed, and satellite moment of inertia. For a circular orbit at 600 km altitude, the satellite speed would be $7.56 km/s$.

\begin{equation}
asd
\end{equation}


\subsubsection{Transformation Between Body \& Reaction Wheel Space}

The main attitude controller sends torque demands to the actuators. The torque demand distributed to the reaction wheels have to be converted to torques parallel to reaction wheel axes, in order for the motor controllers to function. Transformation from reaction wheel space to body frame is quite intuitive. Knowing the orientation, the mounting angle of each motor axis and the corresponding motor torque


\cite{reactionWheelConfigThesis}

\begin{equation}
\vec{N}_{rw} = \underline{A} \vec{N}_{motor} = \begin{bmatrix}
\vec{N}_{1}       & \vec{N}_{2}  & \vec{N}_{3}  & \vec{N}_{4} 
\end{bmatrix} \vec{N}_{motor}
\end{equation}

\begin{equation}
\underline{A} \vec{N}_{motor}  = 
\begin{bmatrix}
\cos(19.47)       & -\cos(19.47) \cos(60)  &  -\cos(19.47) \cos(60)  & 0 \\
0       & \cos(19.47) \cos(30)  &  -\cos(19.47) \cos(30)  & 0 \\
-\sin(19.47)       & -\sin(19.47)   &  -\sin(19.47)   & 1
\end{bmatrix} \vec{N}_{motor}
\end{equation}

Since $A$ is a $ 4 \times 3 $ matrix, a pseudo inverse has to be used when reordering the equation.

\begin{equation}
\vec{N}_{motor} = \underline{A} ^\dagger \vec{N}_{rw}   =  \underline{A}^T  (\underline{A} \underline{A} ^T)^{-1}\vec{N}_{rw}
\end{equation}

\nomenclature{$N_{motor$}}{$4\times1 vector where each entry represents the torque of each reaction wheel motor. $}

\cite[equation 18.41-42]{SADC}
\cite{reactionWheelConfigThesis}



%\begin{equation}
%\vec{N}_{motor} = \underline{A} ^\dagger \vec{N}_{rw} + k [1,-1,-1,1]
%\end{equation}

\subsubsection{Reconfiguration}

Fault isolation for the redundant reaction wheel configuration can be done by detecting which is the reaction wheel where the fault occurred and shutting it off and redistributing the torques to the rest of the reaction wheels. This reconfiguration can be represented by swapping the corresponding faulty columns to zero vectors. For example, if a fault occurs in the 3rd reaction wheel:

\begin{equation}
A_{f3} = \begin{bmatrix}
Axis_{1}       & Axis_{2}  & 0  & Axis_{4} 
\end{bmatrix}
\end{equation}

The pseudo inverse is calculated in the same manner as presented in ... 

\section*{Magnetorquer model}
\nomenclature[Sncoil]{$n_{coil}$}{The number of windings of the coil}
\nomenclature[SIcoil]{$I_{coil}$}{The electric current flowing through the coil}
\nomenclature[SAcoil]{$\vec A_{coil}$}{The vector perpendicular to the cross-sectional area of the magnetorquer}
\nomenclature[Sm]{$\vec m_{mt}$}{The magnetic dipole moment}

Since the primary actuators for the satellite are chosen to be reaction wheels, four magnetorquer will be used for desaturation of the reaction wheels.  

Having a solenoid onboard of the satellite, referred as a magnetorquer through which the current could be controlled and hence the dipole moment.

The interaction of the dipole with the magnetic field of the Earth will result in a torque that will be perpendicular to the magnetic field vector according to the following equation \cite{SADC}:
\begin{flalign}
   \vec N_{mt} = \vec m_{mt} \times \vec B
	\label{eq:NT}
\end{flalign} 
where $\vec N$ is the torque produce by the magnetorquer and will be the torque that will influence the satellite dynamics, $\vec B$ is the vector of the magnetic field of the Earth and $\vec m_{mt} $ is the magnetic dipole moment generated by the magnetorquer.

The magnetic moment $\vec m_{mt}$ is given by \cite{MagMom}:
\begin{flalign}
	\vec m_{mt} = n_{coil} \ I_{coil} \ \vec A_{coil}
	\label{eq:mm}
\end{flalign} 
where $n_{coil}$ is the windings of the coil, $I_{coil}$ is the electric current on the coil and $\vec A_{coil}$ is the vector perpendicular to the cross-sectional area of the magnetorquer.

Using \ref{eq:NT} and \ref{eq:mm} and taking the magnitude, the applied torque on the satellite is \cite{SJ}:
\begin{flalign}
	\vec N_{mt} = n_{coil} \ \rvert I_{coil}\rvert \ \rvert \vec A_{coil}\rvert \ |\vec B| \sin (\theta)
	\label{eq:ft}
\end{flalign} 
where $\sin (\theta)$ is the angle between the plane $A_{coil}$ and the magnetic field vector $\vec B$.

Furthermore, the resistance of the magnetorqer which is a function of the temperature of the coil given as an input, can be computed as
\begin{flalign}
R_{mt} = \dfrac{nC  \rho_{mt} }{A_{wire}} = \dfrac{nC \rho_0(1+\alpha_0(T_{mt} - T_0))}{A_{wire}}
\label{eq:rt}
\end{flalign} 
where \\
$R_{mt}$ is the resistance of the magnetorquer \\
$n$ is the number of windings \\ 
$C$ is the wire circumference  \\
$A_{wire}$ is the wire cross-sectional area  \\
$\rho_0$ is the resistivity of copper  \\
$\alpha_0$ is the coefficient of resistivity temperature   \\
$T_{mt}$ is the temperature given as an input   \\
$T_0$ is the resistivity base temperature  

Using the computed resistance the current is found by dividing the voltage by the resistance of the magnetorquer. Next, in order to find the magnetic moment $m$, the current is multiplied by the number of windings and the area of the wire. For finding the torque that acts on the satellite, the magnetic moment is multiplied by the coil normal and a cross-product is used between this multiplication and the magnetic field of the Earth.

In order to find what voltage to output for having a certain amount of torque, a gain between voltage and magnetic moment is found. For the coil model, the control signal is the voltage. Therefore, translating the magnetic moment demand to voltage is found as follows:
\begin{flalign}
\frac{\vec m_{mt}}{v} = \frac{n_{coil} \vec A_{coil} \vec I_{coil}}{R_{mt}} \mathcal {K}
	\label{eq:gain}
\end{flalign} 
\begin{flalign}
 \mathcal{K} = \frac{\vec m_{mt} R_{mt} }{ n_{coil}  \vec A_{coil} \vec I_{coil} v}
	\label{eq:gainn}
\end{flalign} 
where \\
$v$ is the voltage \\
$\mathcal {K}$ is the gain

The voltage can be found using the following transfer function:
\begin{flalign}
	\frac{I_{coil}}{v} = \frac{1}{R_{mt}}
	\label{eq:voltage}
\end{flalign} 

where the voltage is found to be $1.25 V$

Given a current $I$ going through a coil, the interest is to find the magnetic field $B$ at a certain point located distance $r$ from the coil. Finding an estimate of the magnetic field at any point will give the possibility to change the location of the magnetometers around. For computing the magnetic field in the center of a rectangular coil, the law of Biot-Savart is used \cite{SJ}:
\begin{flalign}
	d\vec B = \frac{\mu_0 I}{4 \pi}  \frac{d \vec s \times \hat{\vec r}}{r^2}
	\label{eq:BS}
\end{flalign} 
where \\
$\vec B$ is the magnetic field \\
$\mu_0$ is a constant called permeability of free space and is equal with $4\pi \times 10^{-7}  \ Tm/A$ \\
$I$ is the current \\
$d \vec s $ is a length element in the direction of current \\
$\hat{\vec r}$ is the direction from $d \vec s$ to a particular position \\
$r$ is the distance from $d \vec s$ to a particular position

The magnetic field $B$ at any point is directly proportional to the current $I$ that is crossing the coil. The magnetic field generated by the current from equation \ref{eq:BS} is just a small length element $d \vec s$ of the coil. For finding the total magnetic field, all small elements need to be summed up and for this the magnetic field $B$ have to be evaluated by integrating equation \ref{eq:BS}.

The design of the magnetorquer is described in appendix \ref{chap:F}.

