\chapter{System Description}\label{chap:systemDescribtion}

\section{Reaction Wheels}

One method of controlling a spacecraft's attitude is by using either reaction or momentum wheels attached to the spacecraft's body. By controlling the wheel's angular velocity using a motor, the amount of angular momentum stored in the wheel can be controlled. If there are no external forces involved, the sum of angular momentum in the system made up by the spacecraft's body and the reaction wheels is constant. This means that by increasing the angular velocity of the wheels, the satellite body's angular momentum can be reduced. This angular momentum transfer can be used to control the attitude of the satellite. If the goal is to change the angular momentum of the whole satellite, actuators that have external interaction should be used, such as magnetorquers or solar sails.

The difference between momentum and reaction wheels is that the nominal angular velocity of momentum wheels is high in order to store angular momentum, and in the case of reaction wheels, low. Many momentum wheels still turn at an angular velocity larger than zero in order to avoid static friction in the bearings. Reaction wheels usually make up only a small fraction of a satellite's weight. They rely on being able to run at high speeds, making their angular momentum significant. The small weight ratio makes precise controlling easier.


%One design consideration for reaction wheels is maximizing moment of inertia for unit weight. This is done by distributing most of the material near the outskirts of the wheel. There is a trade-off between having most of the mass at the outskirts and durability at high angular velocities. 


Reaction wheels have an angular velocity limitation. This means that if a reaction wheel reaches its maximum angular velocity, it can no longer generate a torque on the satellite's body in one direction. In this scenario the system's controllability decreases, thus it should be avoided. An angular momentum unloading strategy should be designed to avoid it. Instead of returning the angular momentum to the satellite's body, unloading the angular momentum through other methods is preferred. Magnetorquers can be used for such purposes.

Moving parts are usually prone to failures. Reaction wheels are expected to run at high angular velocities, which wears down the lubrication and the bearings. Reaction wheels equipped with active magnetic bearings are in development \cite{MagneticReactWheel}. These can eliminate friction from the system and by controlling the bearing, can even reduce micro-vibrations, increasing the durability of the system. AAUSAT-II itself however uses mechanical wheel bearings.

\subsection{Reaction Wheel Configuration}

angular acceleration demand \cite{ReactionWheelConfigSim} \cite{ReactConfigThesis}

gps orientation, Earth station

Tetrahedron configuration can output twice as much force along an axis as one wheel can produce along its own axis.

\subsubsection{Transformation Between Body \& Reaction Wheel Space}

\cite[equation 18.41-42]{SADC}

\begin{equation}
h_{rot} = A\left[ h_1, h_2, h_3, h_4 \right]^T
\end{equation}

\begin{equation}
N = A^R \textbf{$N_c$} + k\left(1,-1,-1,1\right)
\end{equation}

\todo{revise 2nd part}

The torque demand for the satellite's body



\section*{Magnetorqer model}
% The total disturbance torque can now be derived as.... [for disturbances]
% In SD a short description about how many magnetorqers we use and why
% maybe to put the RW and MT in the SD both with general description and model
% ref for magnetorqer: serway and wartz or "Fully magnetic attitude control for spacecraft subject
%to gravity gradient" Rafal
% m is given in the sat body frame and is needed in control frame so we need a rotation 
\nomenclature[S]{$n_{coil}$}{The windings of the coil}
\nomenclature[S]{$I_{coil}$}{The electric current on the coil}
\nomenclature[S]{$\vec A_{coil}$}{The vector perpendicular to the cross-sectional area of the magnetorquer}

Since the primary actuators for the satellite are chosen to be reaction wheels, the magnetorqers will be used for desaturation of the reaction wheels. The satellite contains onboard four magnetorqers mounted perpendicular to each other. 

Having a solenoid onboard of the satellite, referred as a magnetorqer through which the current could be controlled and hence the dipole moment.

The interaction of the dipole with the magnetic field of the Earth will result in a torque that will be perpendicular to the magnetic field vector according to the following equation \cite{SADC}:
\begin{flalign}
   \vec N_{mt} = \vec m \times \vec B
	\label{eq:NT}
\end{flalign} 
where $\vec N$ is the torque produce by the magnetorquer and will be the torque that will influence the satellite dynamics, $\vec B$ is the vector of the magnetic field of the Earth and $\vec m $ is the magnetic dipole moment generated by the magnetorquer.

The magnetic moment $\vec m$ is given by \cite{MagMom}:
\begin{flalign}
	\vec m = n_{coil} \ I_{coil} \ \vec A_{coil}
	\label{eq:mm}
\end{flalign} 
where $n_{coil}$ is the windings of the coil, $I_{coil}$ is the electric current on the coil and $\vec A_{coil}$ is the vector perpendicular to the cross-sectional area of the magnetorquer.

Using \ref{eq:NT} and \ref{eq:mm} and taking the magnitude, the applied torque on the satellite is \cite{SJ}:
\begin{flalign}
	\vec N_{mt} = n_{coil} \ \rvert I_{coil}\rvert \ \rvert \vec A_{coil}\rvert \ |\vec B| \sin (\theta)
	\label{eq:ft}
\end{flalign} 
where $\sin (\theta)$ is the angle between the area $A_{coil}$ and the magnetic field vector $\vec B$.

The design of the magnetorquer is described in appendix \ref{chap:F}.

