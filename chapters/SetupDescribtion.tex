\chapter{System Description}\label{chap:systemDescribtion}

\section{Reference Frames}

\subsubsection{Earth Centered Inertial Frame (ECI)}

\subsubsection{Earth Centered Earth Fixed Frame (ECI)}

\subsubsection{Orbit Frame (ECI)}

\subsubsection{Body Frame (ECI)}

\section{Reaction Wheels}

One method of controlling a spacecraft's attitude is by using either reaction or momentum wheels attached to the spacecraft's body with fixed axes. By controlling the wheel's angular velocity using a motor, the amount of angular momentum stored in the wheel can be controlled. If there are no external forces involved, the sum of angular momentum in the system made up by the spacecraft's body and the reaction wheels is constant. This means that by increasing the angular velocity of the wheels, the satellite body's angular momentum can be reduced. This angular momentum transfer can be used to control the attitude of the satellite.

The difference between momentum and reaction wheels is that the nominal angular velocity of momentum wheels is high in order to store angular momentum, and in the case of momentum wheels, low. Many momentum wheels still turn at an angular velocity larger than zero in order to avoid \todo{static friction, sticking?}.

\todo{find cost of sending 1 kg to space}
One design consideration for momentum wheels is maximizing moment of inertia for unit weight. This is done by distributing most of the material near the outskirts of the wheel. The maximum radius of the wheel is determined by the maximum angular velocity the wheel is designed for. 
Reaction wheels usually make up only a small fraction of a satellite's weight \todo{numbers}. They rely on being able to run at high speeds, making their angular momentum significant. The small weight ration makes precise controlling easier.

saturation

moving part - fault prone

asd

gps orientation, earth station

Tetrahedron configuration can output twice as much force along an axis as one wheel can produce along its own axis.