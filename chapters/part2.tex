\chapter{Fault  analysis}
\textit{In this chapter, probable faults in the system are examined using a \nomenclature{\textbf{FMEA}}{Failure Mode and Effects Analysis} Failure Mode and Effects Analysis (FMEA). Next, in order to find how probable a fault will happen and the effect of it, the severity and occurrance (SO) \nomenclature{\textbf{SO}}{Severity and Occurrance}  of faults is analyzed. It was decided that the fault analysis will be carry out only for the actuators (magnetorquers and momentum wheels).}

A fault in a system can be seen as a sudden shift in the system functionality, nevertheless, it might not mean a total shutdown of the system. One way to see it is as a disturbance in the system, that might cause performance loss or serious deterioration to the system. On the other hand, a failure can be understood as a total shutdown of the system component. 

In \figref{fig:1} a fault tolerant system is shown, which contains an autonomous supervisor that has the ability to switch between various controllers taking into account the type of fault that a component has. The spacecraft block illustrated in the picture is composed of a plant, actuators and sensors and is monitored by the fault detection and isolation \nomenclature{\textbf{FDI}}{Fault Detection and Isolation} (FDI) system, which include detectors that will feed informations to the supervisor in the eventuality of a fault. Based on the information received, the supervisor will establish if a fault occurred or not and in case of a fault the effectors will handle it. Figure \ref{fig:2} shows the procedure of how faults are handled with varius methods. The first step in Fault Analysis is fault modelling which uses a procedure called FMEA.
\begin{table}[H]
	\begin{minipage}[b]{0.49\linewidth}
		\centering
		\begin{figure}[H]
			\centering
			\includegraphics[width=1\linewidth]{figures/FTC}
			\caption{Fault tolerant system architecture [ref jesper article}
			\label{fig:1}
		\end{figure}
	\end{minipage}\hfill
	\begin{minipage}[b]{0.49\linewidth}
		\centering
		\begin{figure}[H]
			\centering
			\includegraphics[width=1\linewidth]{figures/FTC_2}
			\caption{ }
			\label{fig:2}
		\end{figure}
	\end{minipage}
\end{table}
\subsection{Failure Mode and Effects Analysis}
A FMEA analysis which is a bottom-up analysis method is performed for the components of the satellite. The main goal of FMEA is to identify possible faults and their effects on components. In order to evaluate how faults are propagated through the system, a FMEA scheme is constructed.
Another aspect of FMEA analysis is that, the severity of a fault can be determined, which will offer the opportunity to prioritize the faults by severity and in this way focus on the important faults.

In order to control the attitude of the satellite, two types of actuators are used: magnetorquers and reaction wheels. Potential faults are gather into a table which describes the effect and cause, while the satellite is orbiting.
\subsubsection{Magnetorquers}
\begin{table}[H]
	\centering
	\label{my-label}
	\begin{tabular}{|l|l|l|}
		\hline
		\multicolumn{3}{|c|}{\textit{\textbf{Magnetorquers}}}                                          \\ \hline
		\multicolumn{3}{|c|}{Creates a magnetic field that interacts with Earth's magnetic field}                     \\ \hline
		\textbf{Reference} & \textbf{Failure Effect} & \textbf{Failure Cause}                          \\ \hline
		$MT1$                 & Low magnetic field  & \begin{tabular}[c]{@{}l@{}}1) Broken wire or bad soldering\\ 2) Component burned\end{tabular} \\ \hline
		$MT2$                 & Maximum magnetic field power  & Short circuit to the power voltage   \\ \hline
		$MT3$                 & Wrong direction of the magnetic field & \begin{tabular}[c]{@{}l@{}} 1) Misalignment of the magnetorquer\\ 2) Short circuit of some parts of the \\ torquer to the power voltage \end{tabular} \\ \hline
		$MT4$                 & Wrong  power of the magnetic field                 & Floating supplay voltage                                            \\ \hline
	\end{tabular}
	\caption{Potential faults in the magnetorquers}
\end{table}
Description of faults in the magnetorquers: 

$\mathcal{F}_{MT1}$: 
The coil it might have a broken wire or a bad soldering inside, that will lead to a poor generation of magnetic field from the magnetorqers. On the other hand, a component could be burned due to a fluctuation in the current.

$\mathcal{F}_{MT2}$: 
With a short circuit in the power supply, a maximum magnetic field power is expected.

$\mathcal{F}_{MT3}$:
A misalignment of the magnetorquer  due to transportation or a sudden shift during the launch, could affect the direction of the magnetic field.

$\mathcal{F}_{MT4}$:  
A variable supply voltage that could mean a positive or negative operation will end up with an error power of the magnetic field.

The fault vector for the magnetorqers is constructed as follows:

$\mathcal{F}_{MT}$ = [ \ $\mathcal{F}_{MT1}$ \ $\mathcal{F}_{MT2}$ \ $\mathcal{F}_{MT3}$ \ $\mathcal{F}_{MT4}$ ]$^ \mathsf{T}$
\subsubsection{Reaction wheels}
\begin{table}[H]
	\centering
	\label{my-label}
	\begin{tabular}{|l|l|l|}
		\hline
		\multicolumn{3}{|c|}{\textit{\textbf{Reaction wheels}}}                                          \\ \hline
		\multicolumn{3}{|c|}{Produces a torque about the satellite COM in order to rotate it }                     \\ \hline
		\textbf{Reference} & \textbf{Failure Effect} & \textbf{Failure Cause}                          \\ \hline
		$RW1$                & Faulty orientation & \begin{tabular}[c]{@{}l@{}} Shifting of the flywheel  \\   \end{tabular} \\ \hline
		$RW2$                & Unable to control the rotation   & A short-circuit in the power supply  \\ \hline
		$RW3$                & Low power received& \begin{tabular}[c]{@{}l@{}} Short circuit to the ground\\  \end{tabular} \\ \hline
		$RW4$                & No torque received  &  A fault in windings    \\ \hline
	\end{tabular}
	\caption{Potential faults in the reaction wheels}
\end{table}
Description of faults in the reaction wheels: 

$\mathcal{F}_{RW1}$: 
A displacement of the flywheel throughout launch or transportation could result in a error in the orientation that might alter the angular velocity of the satellite.

$\mathcal{F}_{RW2}$: 
A short-circuit in the power supply could influence the rotation of the flywheel, by having a low or maximum rotation, which will end up with a low or maximum power.

$\mathcal{F}_{RW3}$:
A short circuit to the ground due to a borken wire or bad soldering will result in low power generation, that could influence the rotation of the flywheel.

$\mathcal{F}_{RW4}$:  
Due to a fault in the windings, the flywheel will not be able to rotate, therefore, no torque will be received.

The fault vector for the reaction wheels is constructed as follows:

$\mathcal{F}_{RW}$ = [ \ $\mathcal{F}_{RW1}$ \ $\mathcal{F}_{RW2}$ \ $\mathcal{F}_{RW3}$ \ $\mathcal{F}_{RW4}$ ]$^ \mathsf{T}$
\subsubsection{Severity and occurrence evaluation}
The next step into Fault Analysis is the Fault Assessment that involves a Severity Occurrence analysis
To investigate the faults that have the biggest rate of occurrence, the severity of the effects of failures and the probability of occurrence is determined based on the faults from FMEA. This procedure describes how to each fault receives a severity index \nomenclature{\textbf{SO}}{Severity Index}(SO) and an occurrence index \nomenclature{\textbf{OI}}{Occurrence Index} (OI). A table containing the serverity and occurrence for magnetorquers is shown as follows:
\begin{table}[H]
	\centering
	\label{11}
	\begin{tabular}{|l|l|l|l|}
		\hline
		\multicolumn{4}{|c|}{\textit{Magnetorquer}}                                                                                         \\ \hline
		Reference & Severity & Occurrence                                          & SO Index                                      \\ \hline
		$MT1$     & 7        & \begin{tabular}[c]{@{}l@{}} 5\\  4\end{tabular} & \begin{tabular}[c]{@{}l@{}} 35\\ 28\end{tabular} \\ \hline
		$MT2$       & 10       & 3       & 30                \\ \hline
		$MT3$       & 3        & \begin{tabular}[c]{@{}l@{}}2\\ 1\end{tabular} & \begin{tabular}[c]{@{}l@{}}6\\ 3\end{tabular} \\ \hline
		$MT4$       & 4        & 6                   & 24                  \\ \hline
	\end{tabular}
\caption{SO for magnetorquer}
\end{table}
The same procedure is done for momentum wheels as follows:
\begin{table}[H]
	\centering
	\label{12}
	\begin{tabular}{|l|l|l|l|}
		\hline
		\multicolumn{4}{|c|}{\textit{Reaction wheels}}                                                                                         \\ \hline
		Reference & Severity & Occurrence                                          & SO Index                                      \\ \hline
		$RW1$      & 1        & \begin{tabular}[c]{@{}l@{}}7\\ \end{tabular} & \begin{tabular}[c]{@{}l@{}}7\\ \end{tabular} \\ \hline
		$RW2$        & 1        & 4             & 4                                         \\ \hline
		$RW3$        & 2        & \begin{tabular}[c]{@{}l@{}}3\\\end{tabular} & \begin{tabular}[c]{@{}l@{}}8\\ \end{tabular} \\ \hline
		$RW4$        & 4        & 2     & 8               \\ \hline
	\end{tabular}
	\caption{SO for momentum wheels}
\end{table}

The severity index is computed using the following formula:
\begin{flalign}
	SO_{index} = severity \cdot occurrence
	\label{eq:ec1}
\end{flalign} 

\subsubsection{Fault propagation analysis}
In order to observe how faults are propagated through the system and the effect of these faults, a FMEA scheme is constructed. Further computation will be on the appendix.
\subsection{Fault detection and isolation}
%Next step into Fault Analysis is Structual Analysis that is described by Fault Detection and Isolation analysis.

\chapter{Attitude control}
\section{Modelling}
This section describes the mathematical model of the satellite which contains the dynamic and kinematic model, based on the rigid body dynamics and kinematics.
\subsection{Spacecraft dynamic equation}
The satellite dynamics are described using Euler's equation of motion and Newton's laws of motion. 
Using Euler's equation of motion, the relation between the change in angular momentum and the torques that affect the satellite is given as follows:
\begin{flalign}
	\vec{ \dot h} = \vec{N_{ext}} =  \vec{N_{mt}}+ \vec{N_{dist}}
	\label{eq:ec2}
\end{flalign} 
where $h$ is the angular momentum of a rigid body, $N_{ext}$ represent all the external torques that influence the satellite, $N_{mt}$ is the torque from the magnetorquers and $N_{dist}$ is the torque from the disturbances.

The change in angular momentum of the satellite can be express as the product between the angular acceleration and the moment of inertia:
\begin{flalign}
	{\vec{\dot h_{sat}}} = {\underline I_{s}}{\vec{\dot \omega}}
	\label{eq:ec3}
\end{flalign} 
where $h_{sat}$ is the angular momentum of the satellite, $\underline I_{s}$ is the moment of inertia of the satellite and $\vec{\omega}$ is the angular velocity.

Including the momentum wheels, the total angular momentum is given by:
\begin{flalign}
	{\vec{h_{tot}}} = \vec{h_{sat}} + \vec{h_{rw}}
	\label{eq:ec4}
\end{flalign} 
where $\vec{h_{mw}}$ is the angular momentum of the momentum wheels.
Therefore, the total angular momentum is described by:
\begin{flalign}
	{\vec{h_{tot}}} = {\underline I_{s}}{\vec{\omega}}+{\vec{h_{rw}}}
	\label{eq:ec5}
\end{flalign}
By rearranging terms, equation \ref{eq:ec5} becomes:
\begin{flalign}
	{\vec{\omega}} = {\underline I_{s}^{-1}} ({\vec{h_{tot}}}-{\vec{h_{rw}}})
	\label{eq:ec6}
\end{flalign}

Using Euler's equation of motion, the time derivative of $\vec{h_{tot}}$ expressed in the ECI frame is:
\begin{flalign}
	&	\vec{ \dot h_{tot}} = \vec{ \dot h_{sat}} + \vec \omega \times \vec h= \vec{  N_{mt}} + \vec{  N_{dist}} \\
	 &\underline I_s {\vec{\dot{\omega}}} + \vec {\dot{h}_{rw}}+ \vec \omega \times \vec L = \vec{  N_{mt}} + \vec{  N_{dist}} 
	 \label{eq:ec7}
 \end{flalign}
Subsequently, the angular velocity is separtated and expressed as:
\begin{flalign}
{\vec{\dot{\omega}}} = -\underline I_s ^{-1} \vec \omega \times \vec L -\underline I_s ^{-1} \vec {\dot{h}_{rw}} + \underline I_s ^{-1}(\vec{  N_{mt}} + \vec{  N_{dist}}) 
\label{eq:ec8}
\end{flalign}
Next, by replacing the cross product with a skew-symmetric matrix ${\underline S(\vec \omega)}$, \eqref{eq:ec8} becomes:
\begin{flalign}&{\vec{\dot{\omega}}}={-\underline I_{s}^{-1}\underline S(\vec \omega)\underline I_{s}\vec \omega-\underline I_{s}^{-1}\underline S(\vec \omega)\vec h_{rw}-\underline I_s ^{-1}\vec{  N_{rw}} + \underline I_s ^{-1}(\vec{  N_{mt}} + \vec{  N_{dist}})}
\label{eq:ec9}
\end{flalign}
where $N_{mt}$ is the torque from the magnetorquers, $N_{rw}$ is the torque from the momentum wheels and the skew-symmetric matrix is:
\begin{flalign}
	{\underline S(\vec \omega)}
	= 
	\begin{bmatrix}
		0& -\omega_{3}& \omega_{2} \\
		\omega_{3}& 0&-\omega_{1}  \\ 
		-\omega_{2} & \omega_{1} &0
	\end{bmatrix} 
	\label{eq:skewsymmetricmatrix}
\end{flalign}
Moreover, the torque set to the momentum wheels is equal to the time derivative of the angular momentum:
\begin{flalign}
    \vec {N_{rw}} =  {\vec{ \dot{h}_{rw}}}
	\label{eq:ec10}
\end{flalign}
\subsection{Spacecraft kinematic equation}
In this subsection, the focus will be on describing the orientation of the satellite. The method used for describing the satellite attitude is quaternion representation. It was decided to choose quaternion representation, because they provide a way to deal with singularities.

The quaternion $\textbf{q}(t)$ is defined as the attitude quaternion of a rigid body at time $t$ with respect to the inertial frame and at time $t+\Delta t$, the quaternion $\textbf{q}(t+\Delta t)$ is defined. The orientation quaternion can be divided into the quaternion at time $t$ and performing a quaternion multiplication with the rotation in the interval $\Delta t$ as follows:
\begin{flalign}
	\vec{ ^s_iq}(t+\Delta{t}) = \vec{ q}(\Delta {t}) \otimes \vec{ ^s_i q}(t) 
	\label{eq:qp}
\end{flalign}
where the orientation quaternion $	\vec{ ^s_iq}(t+\Delta{t}) $ represents the rotation of the spacecraft body frame with respect to the intertial frame

The quaternion at time $\Delta t$ can be express using the triad $u, v, w$, that represent the axis of the spacecraft as:
%
\begin{flalign}
	q_{1}(\Delta {t})  = {e_{u}\sin\frac{\Delta\Phi}{2}}
	\label{eq:q11}
\end{flalign}
%
\begin{flalign}
	q_{2}(\Delta {t}) = {e_{v}\sin\frac{\Delta\Phi}{2}}
	\label{eq:q2}
\end{flalign}
%
\begin{flalign}
	q_{3} (\Delta {t})= {e_{w}\sin\frac{\Delta\Phi}{2}}
	\label{eq:q3}
\end{flalign}
%
\begin{flalign}
	q_{4}(\Delta {t}) = {\cos\frac{\Delta\Phi}{2}}
	\label{eq:q4}
\end{flalign}
where $\Delta \Phi$ is the rotation at time $\Delta t$ and $e_u,e_v, e_w$ are the components along the triad $u, v, w$ at time $\Delta t$.

Using equation \ref{eq:q11} and equation \ref{eq:q4} and insert them into equation \ref{eq:qp} which yields:
\begin{flalign}
	\vec{ ^s_i q}(t+\Delta{t})
	= 
	\left\{\cos\frac{\Delta\Phi}{2} \underline I_{(4\times4)}+\sin\frac{\Delta\Phi}{2}
	\begin{bmatrix}
		0 &e_{z}&-e_{y}&e_{x} \\
		-e_{z}&0&e_{x}&e_{y}  \\ 
		e_{y}&-e_{x}&0&e_{z} \\
		-e_{x} &e_{y}&-e_{z}&0
	\end{bmatrix} 
	\right \} \vec{ ^s_i q}(t)
	\label{eq:quatm}
\end{flalign}  
%
where $\underline I$ is the identity matrix with the dimensions of $4\times4$.

In order to turn equation \ref{eq:quatm} into a differential equation, a small angle approximation it is used: 
\begin{flalign}
	&\Delta \phi = \omega \ \Delta t \\
	&\cos\frac{\Delta\Phi}{2} \approx 1 \\	
	&\sin\frac{\Delta\Phi}{2} \approx \frac{\omega \Delta t }{2} \\
	\label{eq:aprox}
\end{flalign} 
After using the approximation and substitute the terms into \ref{eq:quatm}, the following equation is obtained:
\begin{flalign}
	\vec{^s_i q(t+\Delta{t})} \approx \left[1 + \frac{1}{2} \underline \Omega \Delta(t)\right]\vec{^s_i q(t)}
	\label{eq:quatfinal}
\end{flalign} 
where $\underline \Omega$ is the skew symmetric matrix written in form:
\begin{flalign}
	\underline \Omega
	= 
	\begin{bmatrix}
		0& \omega_{w}& - \omega_{v}& \omega_{u} \\
		-\omega_{w}& 0&\omega_{u}& \omega_{v}  \\ 
		\omega_{v}& -\omega_{u}&0& \omega_{w} \\
		-\omega_{u}& -\omega_{v}& -\omega_{w}&0
	\end{bmatrix} 
	\label{eq:sm}
\end{flalign}
where the terms $\omega_u, \omega_v, \omega_w$ are the angular velocities componets.

The rate of change in the orientation of the spacecraft $\vec{^s_i q(t)}$  can be found:
\begin{flalign}
	\vec{ ^s_i\dot q(t)} = \lim_{\Delta t\to 0} \frac{\vec q(t+\Delta t) - \vec q(t)}{\Delta t} = \dfrac{1}{2} \underline \Omega \  \vec{^s_i q(t)}
	\label{eq:finaleq}
\end{flalign} 
\subsection{Spacecraft equation of motion }
Putting together both dynamic and kinematic equation for the spacecraft, the system equations can be combined into a state-space representation:
\begin{flalign}
	\begin{bmatrix}
		\vec{ ^s_i\dot q(t)} \\
		\vec{\dot \omega{(t)}}
	\end{bmatrix} 	
	= 
	\begin{bmatrix}
		\frac{1}{2} \underline{ \Omega}_{(4\times4)} \vec{ ^s_i q(t)} \\
		{-\underline{I}_{s}^{-1}\underline{S}(\vec{\omega})\underline{I}_{s}\vec{\omega}(t)-\underline{I}_{s}^{-1}\underline{S}(\vec{\omega})\vec{h_{rw}}-\underline{I}_{s}^{-1}\vec{N_{rw}}(t)+\underline{I}_{s}^{-1}[\vec{N_{mt}(t)}+\vec{N_{dis}}(t)}]
	\end{bmatrix} 
	\label{eq:seom}
\end{flalign}
where,\\
$\vec{ ^s_i  q(t)} = [q_1 \ q_2 \ q_3 \ q_4]^T$ \\
$\vec{\omega{(t)}} = [ \omega_1 \ \omega_2 \ \omega_3]^T$ \\
$\underline{\Omega}(\omega)$ is the $4\times4$ skew symmetric matrix \\
$\underline{I}_{s}$ is the inertia matrix \\
$\underline{S}(\omega)$ is the $3\times3$ skew symmetric matrix \\
$\vec{N_{dis}}(t)$ is the disturbance torque \\
$\vec{N_{rw}}$ is the torque from momentum wheels \\
$\vec{N_{mt}}$ is the torque from magnetorquers  \\

\chapter{Magnetorqer model}
% The total disturbance torque can now be derived as.... [for disturbances]
% In SD a short description about how many magnetorqers we use and why
% maybe to put the RW and MT in the SD both with general description and model
% ref for magnetorqer: serway and wartz or "Fully magnetic attitude control for spacecraft subject
%to gravity gradient" Rafal
% m is given in the sat body frame and is needed in control frame so we need a rotation 

Since the primary actuators for the satellite are chosen to be reaction wheels, the magnetorqers will be used for desaturation of the reaction wheels. The satellite contains onboard four magnetorqers mounted perpendicular to each other. 

Having a solenoid onboard of the satellite, referred as a magnetorqer through which the current could be controlled and hence the dipole moment.

The interaction of the dipole with the magnetic field of the Earth will result in a torque that will be perpendicular to the magnetic field vecotor according to the following equation:
\begin{flalign}
   \vec N_{mt} = \vec m \times \vec B
	\label{eq:NT}
\end{flalign} 
where $\vec N$ is the torque produce by the magnetorquer and will be the torque that will influence the satellite dynamics as stated in ref to dynamic eq ?, $\vec B$ is the vector of the magnetic field of the Earth and $\vec m $ is the magnetic dipole moment generated by the magnetorquer.

The magnetic moment $\vec m$ is given by:
\begin{flalign}
	\vec m = n_{coil} \ I_{coil} \ \vec A_{coil}
	\label{eq:mm}
\end{flalign} 
where $n_{coil}$ is the windings of the coil, $I_{coil}$ is the electric current on the coil and $\vec A_{coil}$ is the vector perpendicular to the cross-sectional area of the magnetorquer.

Using \ref{eq:NT} and \ref{eq:mm} and taking the magnitude, the applied torque on the satellite is [serway]:
\begin{flalign}
	\vec N_{mt} = n_{coil} \ \rvert I_{coil}\rvert \ \rvert \vec A_{coil}\rvert \ |\vec B| \sin (\theta)
	\label{eq:ft}
\end{flalign} 
where $\sin (\theta)$ is the angle between the area $A_{coil}$ and the magnetic field vector $\vec B$.

The design of the magnetorquers placed inside the satellie is inspired from (thesis), where to types of magnetorqers are described: one with metal core and without core. The parameters for one magnetorqer without core are presented in table \ref{table:for}:

\begin{table}[H]
	\centering
	\begin{tabular}{|l|l|}
		\hline
		\textit{\textbf{Parameter}}     & \textit{\textbf{Value}}                     				     \\ \hline
		Coil size                       		   & 75x75 {[}$mm^2$ {]}                      					  \\ \hline
		Wire Thickness                      & 0.13 {[}$mm${]}                             					 	\\ \hline
		Windings                        		& 250                                          						     	 \\ \hline
		Coil mass                     		    & 0.053 [$kg$]                                    					    \\ \hline
		Max voltage                  	      & ±1.25 {[}$V${]} (controlled by PWM duty cycle) 		\\ \hline
		Max current                  	      & 15.78 {[}$mA${]}                               						   \\ \hline
		Actuator on time                   & 88\%                                      				                        \\ \hline
		Max power consumption pr. coil  & 17.4 {[}$mW$ {]}                            					     \\ \hline
		Total power consumption     & 134.2 {[}$mW${]}                           	  					   	    \\ \hline
		Coil Discharge time:             & 0.33481 {[}$ms${]} (99\% discharged)   					     \\ \hline
		Available time for measurements & 11.67 {[}$ms${]}                               					     \\ \hline
	\end{tabular}
	\caption{Parameters for magnetorqer without metal core}
		\label{table:for}
\end{table}

For this type of magnetorqer, one magnetorqer will generate around 200 [$nNm$] at low magnetic field strength (18000 [$nT$), which is perpendicular to the area of the coil. (thesis)

A second alternative is to choose a magnetorqer with metal core, because the power consumption, the size and weight are considered superior compared with a magnetorqer without cores. Moreover, because the interest in redundancy is important and four magnetorqers will be placed inside the satellite, the magnetorqer with metal core is preferable because of their weight and size. The parameters for the magnetorqer with metal core is illustrated in the following table:
\begin{table}[H]
	\centering
	\begin{tabular}{|l|l|}
		\hline
		\textit{\textbf{Parameter}}     & \textit{\textbf{Value}}               \\ \hline
		Core diameter:                       & 10 {[}$mm$ {]}                           \\ \hline
		Core length             			   & 10 {[}$mm${]}                             \\ \hline
		Permeability                           & 1000                                      	     \\ \hline
		Wire Thickness                      & 0.13 [$mm$]                                 \\ \hline
		Windings                                & 200 											      \\ \hline
		Coil mass                               & 0.019 {[}$kg${]}                               \\ \hline
		Max voltage                			  & ±1.25 {[}$V${]}                                  \\ \hline
		Max current							  & 20 {[}$mA$ {]}                                    \\ \hline
		External resistance needed   & 62.5 {[}$\Omega$ {]} 					  	 \\ \hline
		Actuator on time			       & 27 \% {[}$mW${]}                                  \\ \hline
		Max power consumption pr. coil     & 6.75{[}$mW${]} 				           \\ \hline
		Total power consumption		& 35.3 {[}$mW${]}                                      \\ \hline
	\end{tabular}
	\caption{Parameters for magnetorqer with metal core}
	\label{table:for1}
\end{table}
In this case, one magnetorqer will generate around 400 [$nNm$] at low magnetic field strength (18000 [$nT$). Besides the advantage of having a reduced dimension and a better power consumption, the magnetorqer with core, generates a magnetic moment higher than expected. On the other hand, the magnetorqer without core was already tested on the AAUSAT and can be seen as a safe choice of actuator.

\section{Magnetic residual }
%ref: Wertz and for B = 2m/r : Space Mission Analysis and Design
When the satellite orbits the Earth, due to the interference of the magnetic field of the Earth and the satellite magnetic residual, an extra disturbance torque is generated. Because the satellite can not be perfectly isolated, the actuators and sensors will produce a residual magnetic moment. Similarly like magnetorqers, the torque generated by the magnetic residual can be computed using:
\begin{flalign}
	\vec N_{mr} = \vec m \times \vec B
	\label{eq:st}
\end{flalign}
where $\vec m$ is the magnetic residual and $\vec B$ is the magnetic field of the Earth.

The magnetic field of the Earth can be approximated using:
\begin{flalign}
	 B = \dfrac{2M}{R^3}
	\label{eq:ftf}
\end{flalign}
where $M$ is the Earth magnetic moment and $R$ is the radius from Earth center to satellite.
