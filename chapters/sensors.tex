\section{Sensors}

\todo{describe the sensors}

\subsection{Magnetometer}

The orbit of the satellite is predictable, the satellite's location can be described as a function of time. Earth's magnetic field can also be quite accurately modeled. This means that by using magnetorquers, the orientation of the satellite can be approximated by comparing the magnetic field model at the satellite's current location and the direction of the magnetic field measured by the magnetometer. To address the noise in the measurement, including the noise arising from inside the satellite, Kalman or particle filtering, along with sensor fusion can be used, however this is outside of the scope of the thesis.

\subsection{Sun Tracking}

Sun trackers can be much easier to develop than star trackers. It measures the sun's orientation in relation to the satellite frame. By using it alongside other sensors, higher accuracy attitude estimation can be achieved, however when Earth is obstructing the sun rays, it is unable to provide attitude data, thus it is not sufficient to only use a sun tracking sensor.