\chapter{Torque demand source}

In the following chapter will be discussed shortly two attitude controller designs that are based on previous work. The two controllers calculate a torque demand that has to be produced from the motors. The first controller is based on the linearized model \eqref{eq:lele}, and thus is characterized as linear state feedback control, while the second one is based on the non-linear dynamics \eqref{eq:seom} creating a sliding manifold, a hyperplane, such that when the states are on the manifold, will converge to the desired reference. The non-linear controller is called Sliding Mode Control (SMC).\nomenclature[A]{\textbf{SMC}}{Sliding Mode Control}
  
 \subsection{Torque demand based on linear controller}
 
Since the norm of $ \vec{ {\bar{\omega}}} $ is known and equal to the orbital angular velocity ($\approx 0.0011$) comparing this value to $\frac{1}{2}$,becomes small, thus \eqref{eq:lele} can be simplified to 
\begin{flalign}
	\begin{bmatrix}
		\vec{ \dot {\tilde{q}}(t) } \\
		\vec{ \dot {\tilde{\omega}}(t) }
	\end{bmatrix} 	
	= 
	\begin{bmatrix}
		\underline{ 0}_{(3\times3)} &	\frac{1}{2} \underline{\vec 1}_{(3\times3)} \\
		\underline{ 0}_{(3\times3)} &	\underline{ 0}_{(3\times3)}
	\end{bmatrix} 
	\begin{bmatrix}
		\vec{  {\tilde{q}}(t) } \\
		{  {\tilde{\vec \omega}}(t) }
	\end{bmatrix} 	
	-
	\begin{bmatrix}
		\underline{\vec 0}_{(3\times3)} \\
		{\underline I_{s}^{-1}}
	\end{bmatrix} 	
	\vec {\tilde N_{ctrl}}
	\label{eq:lelele}
\end{flalign}
Three equal subsystems can be derived from \eqref{eq:lelele} as
 \begin{flalign}
 	\begin{bmatrix}
 		 \dot { \tilde {q_{i}}} \\
 		 \dot { \tilde { \omega_{i}}}
 	\end{bmatrix} 	
 	= 
 	\begin{bmatrix}
 		 0&	\frac{1}{2}  \\
 		 0 &	 0
 	\end{bmatrix} 
 	\begin{bmatrix}
 		 \tilde{q_{i}}(t)  \\
 		  \tilde{\omega_{i}}(t) 
 	\end{bmatrix} 	
 	-
 	\begin{bmatrix}
 		0 \\
 		 I_{i,s}^{-1}
 	\end{bmatrix} 	
 	\tilde{N_{i}}
 	\label{eq:subsys}
 \end{flalign}
with $i = 1, 2, 3 $. The control torque was defined by the state feedback law as 
\begin{flalign}
	N_{i}	
	= 
	-
	\begin{bmatrix}
	k_{1} &	k_{2} 	
	\end{bmatrix} 
	\begin{bmatrix}
		\tilde{q_{i}}(t)  \\
		\tilde{\omega_{i}}(t) 
	\end{bmatrix} 	
	\label{eq:inputtorque}
\end{flalign}
leading to a second order closed loop system calculated as $det(s\underline{I} - (\underline{A} - \underline{BK}) )$. Identifying  this with a general second order equation $s^{2}+2\zeta\omega_{n}s+\omega_{n}^{2}$, with $\zeta$ be the dumping factor which was chosen to be equal to 1 leading to an over dumped response and $\omega_{n}$  the natural frequency $\omega_{n} =  \frac{2\pi}{60/0.35} $ with 60 be the value of the chosen rise time, the controller gains was derived as

\begin{flalign*}
k_{1} = -2 I_{i,s} \omega_{n}^{2} 
\label{eq:gainsl22}
\end{flalign*}
\begin{flalign*}
k_{2} = -2\zeta I_{i,s} \omega_{n}^{2} 
\label{eq:gainsl223}
\end{flalign*}
which give stability for the all values of $ \vec{ {\bar{\omega}}} $. 
\subsection{Torque demand based on sliding mode controller}

As described previously the sliding mode control scheme belongs to the class of non-linear control designs and is more robust compared to the linear, when disturbances are present. The objective of the SMC is the design, from a geometrical point of view, of a hyperplane, commonly called manifold, in the state space which whenever the states are on the manifold, the behavior of the system will meet the specifications it is designed for, i.e convergence to the desired reference. \nomenclature[S]{\textbf{i.e}}{id est : in other words}   
 Introducing the small signal deviation of the states as
\begin{flalign}
\vec{\tilde{q}} = \vec{  \bar{q}}^{-1} \otimes \vec{ q} 
\label{eq:smallsignal22}
\end{flalign}
for the quaternion error and with $\vec{  \bar{q}}^{-1}$ be the reference quaternion and $\vec{ q} $ be the measured, and for the angular velocity
\begin{flalign}
\vec{\tilde{\omega}}  = \vec{\omega}-\vec{\bar{\omega}}  
\label{eq:smallsi4gnal4566}
\end{flalign}
with $\vec{\bar{\omega}}$ be the nominal value of the angular velocity. Moreover, by introducing the sliding variable as $s$ which depends on the error \nomenclature[S]{\textbf{s}}{Sliding variable} and is chosen to be 

\begin{flalign}
s  = F\vec{\tilde{q}} + \vec{\tilde{\omega}}  
\label{eq:sliding variable}
\end{flalign}
with $F = \alpha\ast diag[111]$ be a positive definite matrix. For $s=0$, $\vec{\tilde{\omega}} = - \alpha\ast\vec{\tilde{q}_{1:3}}$ and thus $\alpha$ can be designed appropriately to give the desired convergence for $\vec{\tilde{q}}$ as $[0;0;0;1]$, more details can be found in \todo{create appendix}   