\subsection{Luenberger Observer} \label{sec:simpleObserver}

Parallel to structural analysis methods also two observer based methods have been designed for fault detection in magnetorquer based actuators. Here will be discussed a Luenberger-like form which is based directly on the non-linear dynamics. In previous chapter has been discussed that for each axis it is assumed a pair of magnetorquers. The redundant magnetic actuators are used for reconfiguration in the presence of a fault or failure.\\    The Luenberger-like observer is based on the dynamic  \eqref{eq:seom} and for the sake of brevity is rewritten here as   
%
\begin{flalign}
\vec{\dot \omega}	
= 
{-\underline{I}_{s}^{-1} \underline{\omega}^\times \underline{I}_{s}\vec{\omega}-\underline{I}_{s}^{-1}\underline{\omega}^\times \vec{h_{rw}}+\underline{I}_{s}^{-1}[\vec{N_{rw}} + \vec{N_{mt}}+\vec{N_{dist}}}]
\label{eq:seom22}
\end{flalign}
%
where the time dependency of the variables has been suppressed for clarity. Rearranging the above equation and by adding the fault vector $\vec{f_{act}}$ it can obtained 
%
\begin{flalign}
\vec{\dot \omega}	
= 
\underbrace{-\underline{I}_{s}^{-1}  [(\underline{I}_{s}\vec{\omega})^\times- (\vec{h_{rw}})^\times]}_{\underline{A}}\vec{\omega}+\underbrace{\underline{I}_{s}^{-1}}_{\underline{B}} \underbrace{[\vec{N_{rw}} + \vec{N_{mt}}]}_{\vec{u}}+\underline{I}_{s}^{-1}\vec{N_{dist}}+\vec{f_{act}}
\label{eq:seom2244}
\end{flalign}
%  
where $\underline{A}=-\underline{I}_{s}^{-1}  [(\underline{I}_{s}\vec{\omega})^\times- (\vec{h_{rw}})^\times] $ is the system matrix, $ \underline{B}= \underline{I}_{s}^{-1}$ is the input matrix, $\vec{u} =\vec{N_{rw}} + \vec{N_{mt}} $ is the input vector and $\vec{N_{dist}}= \vec{d}$ is the disturbance vector. The system can now be written in Luenberger-like form as
%
\begin{flalign}
\vec{\hat{{\dot \omega}}} = \underline{A}{\hat{{\vec{\omega}}}}+\underline{B}\vec{u}+\underline{B}\vec{d}+\underline{L}\underline{C}({\vec{\omega}}-{\hat{{\vec{\omega}}}})
\label{eq:seom255554}
\end{flalign}
% 
with $\underline{L}$ be the observer gain and the output vector can be written as
\begin{flalign}
\vec{y} = \underline{C}\vec{\omega}
\label{eq:seom2554}
\end{flalign}
with $\underline{C}$ be identity matrix. The matrix $\underline{A}$ is found by using the maximum values of $\vec{\omega}$ and $\vec{h_{rw}}$ which were obtained by running the simulation over one period, and thus the gain matrix is obtained by pole placement as
\begin{equation}
\underline{A}_{M,orth}  = 
\begin{bmatrix}
-3.0000       & -0.0014 &  0.0004 \\
0.0011       &-4.0000  &  -0.0163  \\
-0.0003    &  0.0163   & -5.0000
\end{bmatrix} 
\label{eq:orthoMatrix22}
\end{equation}