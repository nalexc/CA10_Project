\chapter{Closure}

\section{Results}

A satellite attitude control scheme for two cooperating actuator subsystems was implemented in a simulation environment for a fault-free system. The two subsystems supplement each other by eliminating each other's weaknesses. The reaction wheel is capable of arbitrary torque exertion at any given moment, however is susceptible to saturation, the magnetorquer can only exert torques in 2 dimensions at any given moment, but is capable of desaturating the reaction wheels. The attitude control system was designed to be modular in order to be able to implement parallel control loops.
Nadir pointing capability of these control methods were proven. The control scheme can also satisfy the more difficult problem of Earth station tracking. Earth station tracking is a benchmark of great attitude control ability. The simulation environment included environmental disturbances, but signal disturbances were not considered. 

The system was designed to be fault-tolerant, keeping satisfactory controllability even when actuator faults occur. In order to implement a fault-tolerant control scheme, the problem of fault detection had to be addressed. It was shown that certain type of faults can be detected using methods that have low computation requirements. Detecting other faults, such as axis misalignment proved to be more problematic. Non-observer based fault detection methods in real-life applications require filtering, which is out of the scope of the thesis. Using observers eliminate the need for filtering, thus if an observer works in the simulation environment neglecting signal disturbances, their real-life implementation can prove successful with a higher probability. Experiments were made using unknown input observer based fault detection. It proved useful for detecting fault in attitude estimation. It was shown however that the using UIO for actuator fault detection is problematic due to the fact that environmental disturbances have similar effect as actuator faults.

Based on fault detector signals, faults have been handled by shutting down the adequate member of the redundant actuator subsystem and redistributing the control demand for the functioning actuators. For reaction wheels, smooth reconfiguration was guaranteed using special transition controllers. Reaction wheel reconfiguration resulted in increased power consumption for certain maneuvers, but the reconfigured control system still managed to satisfy the control requirements.


%power usage increase

\section{Discussion}

State of the art control schemes have been implemented and combined in the simulation environment, adjusted to the specific requirements of fault tolerant control.

The limitation of unknown input observers has been reached with the torque disturbance not being distinguishable from actuator fault. Literature research has shown how observer design becomes significantly more difficult for nonlinear systems.

The thesis has shown that exact identification of the fault source is not required to perform fault tolerant control for redundant actuator system, isolating the actuator where the fault occurred proved sufficient. This relaxes the amount of 'detective work' required in the system.

\section{Conclusion}
In this thesis actuator fault detection and fault tolerant control was examined for a pico-satellie. For fulfilling the requirements, various control schemes were developed and tested in a simulation environment. The simulated satellite was able to accommodate faults autonomously.

%Attitude control is achieved by using three different attitude controlers. 

