\chapter{Requirements}\label{chap:requirements}
Based on the use-case introduced and the available system a set of requirements are formulated.
%
\subsection*{System requirements}
%
\begin{enumerate}
	\item \textbf{The satellite should reconfig scheme control .}
	
	desc
	
	\item \textbf{The satellite should detect the faults .}
	
	desc
	
\end{enumerate}

MUST is equivalent to REQUIRED and SHALL indicating that the definition is an absolute requirement.

MUST NOT is equivalent to SHALL NOT and indicates that it is an absolute prohibition of the specs.

SHOULD is equivalent to RECOMMENDED means that there are valid reasons to ignore a particular requirement, but the implications need to be weighed.

SHOULD NOT and NOT RECOMMENDED means that a particular behavior may be acceptable or useful, but again, the implications need to be weighed.

MAY means OPTIONAL and that the requirement is truly optional. Interoperability with different systems that may or may not implement an optional requirement must be done.

$\dot \omega $