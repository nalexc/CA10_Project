\subsubsection{Gradient anomaly detection }
\label{sec:ModelFreeFD}

Some faults can cause anomalies in signals that are easy to notice by simply analyzing the signal shape they produce. Discontinuities for example often hint at a fault. Since most signals are sampled discretely, discontinuities appear as large signal steps. For example, if the angular velocity sensor in the reaction wheel motors fall to zero instantaneously, there's a good chance that a sensor fault has just occurred. Detecting faults through such anomalies require no knowledge of the system dynamics, thus they can be deemed as model free fault detection.

A reconfiguration scheme using anomaly detection in angular velocity sensors has been implemented in the simulation environment. The detection checks the magnitude of the angular velocity gradient, and if it's above the threshold, the fault detector module sends a fault signal to the reaction wheel supervisor system, which can decide to reconfigure the reaction wheel torque distribution. The signal equation is given by equation \ref{eq:modelFreeDetect}, where $\dot{\omega}_{thresh}$ denotes the threshold for fault signal generation, $\Delta t$ denotes the sampling time.  $\dot{\omega}_{thresh}$ can be set according to maximum reaction wheel torque, maximum reaction wheel angular velocity and wheel friction. $\dot{\omega}_{thresh}$ should at  least be $\dot{\omega}_{thresh} > \frac{b \omega_{max} + \tau_{max}}{J_{motor}}$ if false fault signals are to be avoided.

\begin{equation}
\label{eq:modelFreeDetect}
FaultSignal_i = |\omega_{w,i}(k) - \omega_{w,i}(k-1)| > \dot{\omega}_{thresh} \Delta t
\end{equation}

\subsubsection{Verify the detectability of potential faults}
The faults are verified according to the FMEA analysis.

By taking a measurement of the voltage indicates that the fault $\mathcal{F}_{RW2}$ is not possible to detect due to structure analysis, because the structure analysis is checking the relation between the measured quantities, which are done according to the model. Therefore, if the measured quantities are wrong then the fault can not be detected.  Having a short-circuit in the power supply denote that the control voltage is zero all the time. It represents a simple model free detection by knowning the saturation, torque demand and the voltage measurement.

The fault corresponding to the faulty orientation $\mathcal{F}_{RW1}$ can be detected. This is possible due to the fact that the structural analysis for the overall dynamics includes the assumption that the orientation are similar with the model, but if the actual orientation changes then the fault can be detected. Moreover, if the torque demand for the reaction wheel is non zero and the wheel is not in saturation, but the reaction wheel still does not generate any output torque, then there is a fault.

The fault equivalent to the bearings fail $\mathcal{F}_{RW5}$ can also be detected. In case a bearing has a fault, then the friction in the residual \ref{eq:res1} changes, therefore it will be detected as a fault.


