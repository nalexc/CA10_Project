

\section{Magnetorqer model}
% The total disturbance torque can now be derived as.... [for disturbances]
% In SD a short description about how many magnetorqers we use and why
% maybe to put the RW and MT in the SD both with general description and model
% ref for magnetorqer: serway and wartz or "Fully magnetic attitude control for spacecraft subject
%to gravity gradient" Rafal
% m is given in the sat body frame and is needed in control frame so we need a rotation 
\nomenclature[S]{$n_{coil}$}{The windings of the coil}
\nomenclature[S]{$I_{coil}$}{The electric current on the coil}
\nomenclature[S]{$\vec A_{coil}$}{The vector perpendicular to the cross-sectional area of the magnetorquer}

Since the primary actuators for the satellite are chosen to be reaction wheels, the magnetorqers will be used for desaturation of the reaction wheels. The satellite contains onboard four magnetorqers mounted perpendicular to each other. 

Having a solenoid onboard of the satellite, referred as a magnetorqer through which the current could be controlled and hence the dipole moment.

The interaction of the dipole with the magnetic field of the Earth will result in a torque that will be perpendicular to the magnetic field vector according to the following equation \cite{SADC}:
\begin{flalign}
   \vec N_{mt} = \vec m \times \vec B
	\label{eq:NT}
\end{flalign} 
where $\vec N$ is the torque produce by the magnetorquer and will be the torque that will influence the satellite dynamics, $\vec B$ is the vector of the magnetic field of the Earth and $\vec m $ is the magnetic dipole moment generated by the magnetorquer.

The magnetic moment $\vec m$ is given by \cite{MagMom}:
\begin{flalign}
	\vec m = n_{coil} \ I_{coil} \ \vec A_{coil}
	\label{eq:mm}
\end{flalign} 
where $n_{coil}$ is the windings of the coil, $I_{coil}$ is the electric current on the coil and $\vec A_{coil}$ is the vector perpendicular to the cross-sectional area of the magnetorquer.

Using \ref{eq:NT} and \ref{eq:mm} and taking the magnitude, the applied torque on the satellite is \cite{SJ}:
\begin{flalign}
	\vec N_{mt} = n_{coil} \ \rvert I_{coil}\rvert \ \rvert \vec A_{coil}\rvert \ |\vec B| \sin (\theta)
	\label{eq:ft}
\end{flalign} 
where $\sin (\theta)$ is the angle between the area $A_{coil}$ and the magnetic field vector $\vec B$.

The design of the magnetorquers placed inside the satellie is inspired from \cite{TH}, where to types of magnetorqers are described: one with metal core and without core. The parameters for one magnetorqer without core are presented in table \ref{table:for}:

\begin{table}[H]
	\centering
	\begin{tabular}{|l|l|}
		\hline
		\textit{\textbf{Parameter}}     & \textit{\textbf{Value}}                     				     \\ \hline
		Coil size                       		   & 75x75 {[}$mm^2$ {]}                      					  \\ \hline
		Wire Thickness                      & 0.13 {[}$mm${]}                             					 	\\ \hline
		Windings                        		& 250                                          						     	 \\ \hline
		Coil mass                     		    & 0.053 [$kg$]                                    					    \\ \hline
		Max voltage                  	      & ±1.25 {[}$V${]} (controlled by PWM duty cycle) 		\\ \hline
		Max current                  	      & 15.78 {[}$mA${]}                               						   \\ \hline
		Actuator on time                   & 88\%                                      				                        \\ \hline
		Max power consumption pr. coil  & 17.4 {[}$mW$ {]}                            					     \\ \hline
		Total power consumption     & 134.2 {[}$mW${]}                           	  					   	    \\ \hline
		Coil Discharge time:             & 0.33481 {[}$ms${]} (99\% discharged)   					     \\ \hline
		Available time for measurements & 11.67 {[}$ms${]}                               					     \\ \hline
	\end{tabular}
	\caption{Parameters for magnetorqer without metal core}
		\label{table:for}
\end{table}

For this type of magnetorqer, one magnetorqer will generate around 200 [$nNm$] at low magnetic field strength (18000 [$nT$), which is perpendicular to the area of the coil. (thesis)

A second alternative is to choose a magnetorqer with metal core, because the power consumption, the size and weight are considered superior compared with a magnetorqer without cores. Moreover, because the interest in redundancy is important and four magnetorqers will be placed inside the satellite, the magnetorqer with metal core is preferable because of their weight and size. The parameters for the magnetorqer with metal core is illustrated in the following table:
\begin{table}[H]
	\centering
	\begin{tabular}{|l|l|}
		\hline
		\textit{\textbf{Parameter}}     & \textit{\textbf{Value}}               \\ \hline
		Core diameter:                       & 10 {[}$mm$ {]}                           \\ \hline
		Core length             			   & 10 {[}$mm${]}                             \\ \hline
		Permeability                           & 1000                                      	     \\ \hline
		Wire Thickness                      & 0.13 [$mm$]                                 \\ \hline
		Windings                                & 200 											      \\ \hline
		Coil mass                               & 0.019 {[}$kg${]}                               \\ \hline
		Max voltage                			  & ±1.25 {[}$V${]}                                  \\ \hline
		Max current							  & 20 {[}$mA$ {]}                                    \\ \hline
		External resistance needed   & 62.5 {[}$\Omega$ {]} 					  	 \\ \hline
		Actuator on time			       & 27 \% {[}$mW${]}                                  \\ \hline
		Max power consumption pr. coil     & 6.75{[}$mW${]} 				           \\ \hline
		Total power consumption		& 35.3 {[}$mW${]}                                      \\ \hline
	\end{tabular}
	\caption{Parameters for magnetorqer with metal core}
	\label{table:for1}
\end{table}
In this case, one magnetorqer will generate around 400 [$nNm$] at low magnetic field strength (18000 [$nT$). Besides the advantage of having a reduced dimension and a better power consumption, the magnetorqer with core, generates a magnetic moment higher than expected. On the other hand, the magnetorqer without core was already tested on the AAUSAT and can be seen as a safe choice of actuator.


