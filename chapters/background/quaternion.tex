\section{Quaternions} \todo{put into appendix}
\todo{add q multiplication, conjugate, rotation, conjugate, separately say that eta is cos fi2}

There are several possible mathematical representations for rotation. In physics, rotation matrices, Euler angles (eg. pitch-roll-yaw) and quaternions. In satellite engineering, quaternions are the preferred representations, since they are more compact than rotation matrices and lack singularities.

Quaternions include four values, three of them represent a vector \textbf{$\epsilon$}, the fourth a scalar $\eta$. 

\begin{equation}
\textbf{q} =
\left[ 
  \begin{array}{cccc}
  q_1 \\
  q_2 \\  
  q_3 \\
  q_4 
  \end{array}
\right] 
= 
\left[ 
\begin{array}{cccc}
\textbf{$\epsilon$} \\
\eta
\end{array}
\right] 
\end{equation}

Let \textbf{e} represent the unit length rotation axis, with \textbf{i}, \textbf{j}, \textbf{k} being the base vectors in euclidean space. 

\begin{equation}
\textbf{e} = e_1 \textbf{i}+ e_2 \textbf{j} + e_3 \textbf{k}
\end{equation}

A rotation with $\Phi$ around the unit vector can be described according to Euler's formula.

\begin{equation}
	q = e^{\frac{\Phi}{2} (e_1 \textbf{i}+ e_2 \textbf{j} + e_3 \textbf{k})} = \cos \frac{\Phi}{2} + (e_1 \textbf{i}+ e_2 \textbf{j} + e_3 \textbf{k}) \sin \frac{\Phi}{2}
\end{equation}

Consequently 

\begin{equation}
	\textbf{q} =
	\left[ 
	\begin{array}{cccc}
	q_1 \\
	q_2 \\  
	q_3 \\
	q_4 
	\end{array}
	\right] 
	= 
	\left[ 
	\begin{array}{cccc}
	e_1  \sin \frac{\Phi}{2} \\
	e_2  \sin \frac{\Phi}{2} \\  
	e_3  \sin \frac{\Phi}{2} \\
	\cos \frac{\Phi}{2} 
	\end{array}
	\right] 
\end{equation}
