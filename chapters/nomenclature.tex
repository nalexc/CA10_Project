\section*{Notations}

Vectors used have a bold typeface.  
\begin{equation*}
\textbf{v}
\end{equation*}
Matrices are underlined.
\begin{equation*}
\underline{A}
\end{equation*}

Cross product operations can be evaluated by taking the skew symmetric matrix of the left vector and executing a matrix multiplication. The skew symmetric matrix of $\textbf{v}$ is denoted as $\underline{v}^\times$
\begin{equation*}
	\textbf{w} = \textbf{u} \times \textbf{v} = \underline{u}^\times \textbf{v}
\end{equation*}
Matrix transposition is denoted as
\begin{equation*}
\underline{A}^T
\end{equation*}
If a non-square matrix has to undergo an operation similar to inversion, Moore-Penrose pseudoinverse is used. Pseudoinverse matrix is indicated as $\underline{A}^\dagger$. If the matrix satisfies $rank(\underline{A}) = min(m,n)$ and $m < n$, the left pseudoinverse is used as follows
\begin{equation*}
\underline{A}^\dagger    =   (\underline{A}^T \underline{A} )^{-1} \underline{A}^T 
\end{equation*}

If $n < m$, the right pseudoinverse is used as follows

\begin{equation*}
 \underline{A}^\dagger    =  \underline{A}^T  (\underline{A} \underline{A}^T)^{-1}
\end{equation*}


The majority of equations are expressed in body-fixed frame (BFF). Unless it's not explicitly noted, the matrices and vectors are expressed in BFF. In case the expression is in earth centered inertial frame (ECI), it is noted in the superscript as 

\begin{equation*}
\vec{XY}^{[I]}
\end{equation*}

The rotation quaternion between frames use the subscript to denote the frame where the transformation is done from, the superscript is the symbol of the frame being transformed into. In case the frame symbols are not present, it should be interpreted as a transformation from inertial frame to body frame.

\begin{equation*}
\vec{^s_i q(t)}
\end{equation*}


Rotation matrix corresponding to rotation matrix $\vec{^s_i q(t)}$ is denoted as
\begin{equation*}
\underline{R}(\vec{^s_i q(t)})
\end{equation*}

Moreover when only the vector part of the quaternion is used is denoted as 
\begin{equation*}
\vec{q}_{1:3}
\end{equation*}

Description of faults in magnetorquers or reaction wheels

\begin{equation*}
\mathcal{F}_{MT} \ or  \  \mathcal{F}_{RW}
\end{equation*}

The complex conjugate is defined

\begin{equation*}
 \vec q^\ast
\end{equation*}
\todo{move last part}

