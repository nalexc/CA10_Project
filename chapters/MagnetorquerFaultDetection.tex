
\subsection{Magnetorquer  fault detection}
%For magnetorquers fault detection, two types of structural analysis were implemented. First, a locally structure analysis for each magnetorquer was implemented, and second a second a structure analysis using the satellite dynamics.
Model based methods have been used for fault detection in magnetorquer actuators. Structural analysis based methods will be discussed first that have been implemented locally in the magnetorquer system and subsequently two observer based designs. 
\subsection{Magnetorquer local structural analysis} \label{sec: MTStructAnal}
Using the Biot-Savar law from equation \ref{eq:BS}, the residual for the magnetorquer is generated using the following equation:
\begin{flalign}
	residual = B_{mt} - \frac{4 \mu_0}{\sqrt{2} \pi L} I
	\label{eq:BfS}
\end{flalign} 
where \\
$ B_{mt}$ is the magnetic field of the magnetorquer\\
$\mu_0$ is a constant called permeability of free space \\
$I$ is the current \\
$L$ is the length of the coil

The Biot-Savart is applied for the magnetometer located at the center of the square coil.

The magnetometers measure the combined magnetic field from Earth and all the magnetorquers, the magnetic field generated by the examined magnetorquer has to be decoupled through sensor fusion.


